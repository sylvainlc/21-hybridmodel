\documentclass[10pt]{article} % For LaTeX2e
\usepackage{tmlr}
% If accepted, instead use the following line for the camera-ready submission:
%\usepackage[accepted]{tmlr}
% To de-anonymize and remove mentions to TMLR (for example for posting to preprint servers), instead use the following:
%\usepackage[preprint]{tmlr}

% Optional math commands from https://github.com/goodfeli/dlbook_notation.
\input{math_commands.tex}

\usepackage{hyperref}
\usepackage{url}
\usepackage{amssymb,amsmath,amsthm}
\usepackage[pdftex]{graphicx}

\title{HERMES: Hybrid Error-corrector Model with inclusion of External Signals for nonstationary fashion time series}

% Authors must not appear in the submitted version. They should be hidden
% as long as the tmlr package is used without the [accepted] or [preprint] options.
% Non-anonymous submissions will be rejected without review.

\author{\name \'Etienne David \email etienne.david@heuritech.com \\
      \addr SAMOVAR, Télécom SudParis,\\
      Institut Polytechnique de Paris, 91120 Palaiseau, France
      \AND
      \name Jean Bellot \email jean.bellot@heuritech.com \\
      \addr Heuritech, \\
      6 Rue de Braque, 75003 Paris
      \AND
      \name Sylvain Le Corff \email sylvain.lecorff@gmail.com\\
      \addr LPSM, \\
      Sorbonne Université, UMR CNRS 8001, 75005, Paris
      }

% The \author macro works with any number of authors. Use \AND 
% to separate the names and addresses of multiple authors.

\newcommand{\fix}{\marginpar{FIX}}
\newcommand{\new}{\marginpar{NEW}}

\def\month{01}  % Insert correct month for camera-ready version
\def\year{2023} % Insert correct year for camera-ready version
\def\openreview{\url{https://openreview.net/forum?id=XXXX}} % Insert correct link to OpenReview for camera-ready version


\begin{document}


\maketitle

We would like to thank the reviewer for the appreciation of the paper and the constructive feedback. We have carefully considered all comments and we will propose a revision of the paper accordingly. In the meantime, you can find below detailed responses and an overview of some of the  modifications that will be added in the revised manuscript.

\section*{Detailed responses}

\subsection*{Requested changes:}
\begin{itemize}
	\item {\em ``The reliability of the data is important for the use of this data by third parties. Please explain in detail how the human processing was done. Also, to what extent can machine failures be included in image processing ? Do humans correct them ?''} \medskip
	
	\textbf{Author response:} We thank the referee for this remark. Regarding the data-acquisition pipeline presented in Section 2, the human actions are limited. The only human processing comes at the step where the fashion trends are defined. The features are automatically detected by the vision recognition models. Then, we only define trends that are relevant for the fashion industry. For instance, the Mini A-line dress fashion trend is identified where the dress category is detected with a particular shape and a particular length, all of these details detected with different models. These associations of fine-grain attributes are defined by a team of fashion experts and are not provided with the paper. 
Concerning the machine failures, all the visual recognition models used to build the proposed dataset have been trained with a standard process involving a train, a eval and a test set. Thus, models showing poor performance in terms of accuracy and recall on the test set have been dismissed. Please note that the data acquisition pipeline has been detailed and extended in Section 2. \\
	
	\item {\em ``In publishing the data, it would be good to add a little more analysis of the characteristics of the data. For example, basic statistics for each market, statistics for the Weak signal $w^{f,i}_t$, etc.''} \medskip
	
	\textbf{Author response:} We agree with the referee and we acknowledge that we have not sufficiently presented the diversity of behaviours hidden in the proposed fashion dataset. So as to improve this point, a new section is added in appendix of the revised version of the paper where 4 different sub-samples of time series are defined and studied separately. \\
	
	\item {\em ``It is written that represents the index of fashion trends, but it is unclear what exactly it refers to. Does it correspond to a product/item?''} \medskip

	\textbf{Author response:} We understand this remark and further information has been added in the revised paper to clarify this point. The index of fashion trends represents the name of a fashion clothing relevant for the fashion industry. It can be just one attribute detected by a visual recognition model (ex: the Sneakers) but it can also be a combination of several attributes (ex: the Slim Sole Retro Basketball Sneakers). As the fashion trends names are anonymized in the proposed dataset, the architecture behind the fashion trends definition is not presented in this paper.\\
	
	\item {\em ``I didn't know what statistic was $\hat{y}^{c,g,m}$. Could you please explain it in detail?''} \medskip

	\textbf{Author response:} We are not sure of the notation in the question of the referee, in the paper we introduce $\bar{y}^{c,g,m}$ and $\Tilde{y}^{c,g,m}$ but not $\hat{y}^{c,g,m}$. We present below a explanation of the two statistic $\bar{y}^{c,g,m}$ and $\Tilde{y}^{c,g,m}$:
	\begin{itemize}
	\item $\Tilde{y}^{c,g,m}$ is the time series representing the behaviour of the global cloth type $c$ for the gender $g$ on the market $m$. It can be for example the time series representing the evolution of the general pants for females in Europe. The sequence $\Tilde{y}^{c,g,m}$ is then used to normalize each time series $y^{c,g,m,i}$ with the same $c$, $g$ and $m$. All the fashion trends of pants for females in Europe are normalized by the time series representing the general pants for females in Europe.
	\item $\bar{y}^{c,g,m}$ is the time series $\Tilde{y}^{c,g,m}$ where the seasonality component has been removed. The seasonality component is removed as we only want to remove social media bias with the normalization step and not remove or change the seasonality of the proposed fashion trends.
	\end{itemize}	
To remove all gray area on this topic, additional information (regarding the Jersey Top normalization example) has been added at the end of Section 2.2 and in the legend to Figure 3.

	\item {\em ``It may be a good idea to modify the structure of Section 3. It is better to explain the hybrid model (4) at first.  Then, elaborate on the motivation behind the formulation of (4). The output of the RNN appears to represent the weights in summing $f^n(.)$ and $\bar{y}^n_{T}$. Is that correct? Also, when is one of $y^{pred}$ and $y^{corr}$ more dominant than the other? The cost function described in Section 4.1 should be moved to Section 3.''} \medskip

	\textbf{Author response:} We thank the referee for this remark. In the proposed method, $f^n(.)$ represents a time-series-specific parametric model, the model $f^n(y^n_{1:T},\theta^n_{predictor})$ is linked to the time series $y^n_{1:T}$. By contrast, the recurrent neural network denoted $RNN()$ in the paper is global: the same network is used for every time series. It is why we introduce $\bar{y}^n_T$ to rescale inputs and outputs of $RNN()$ as we do not assume that all the time series have the same volume. The fraction of the final prediction due to $y^{pred}$ or $y^{corr}$ is variable and depends on the nature of the time series (and the external signal if it is provided to the corrector model). A good illustration of this point is presented in a new section added in the appendix of the revised paper. In this section, we study the accuracy of the HERMES model on sub-samples of time series sharing the same behaviour. On the sub-sample of time series with a high level of noise, predictions of the HERMES model seem to rely only on $y^{pred}$. By contrast, on a sub-sample called 'disrupting trend', we can see that $y^{corr}$ strongly impacts the final prediction resulting in an improvement of the final accuracy of the HERMES model. Concerning the cost function described in Section 4.1, we decided to present this part in the Experiments section and not in the section presenting the model as the choice of the loss function can change depending on the use case. For instance, the L1 loss function was selected to train the HERMES variations on the Fashion dataset while the MASE loss has been used for the M4 dataset.\\	

	\item {\em ``It would be good to add a discussion of the limitations of this study. Other external factors could be various, such as advertisements. This study focuses on the correlation between weak signals and trends, not causality. This should be clearly stated. As seen in Figure 2, the effect of the external factor is accompanied by a time delay. It would be interesting to analyze how such time delays vary by fashion category and market. It would also be interesting to incorporate this into machine learning models.''} \medskip

	\textbf{Author response:} We thank the referee for this remark and supplementary information has been added in the result sections and in the appendix of the revised paper. As future works on the dataset, we attempt to build a second fashion dataset gathering more time series and additional external signals such as fashion shows signals, brands signals etc...
Concerning the forecasting model, we acknowledge that we did not discuss enough about the limits of the proposed approach. In the revised version of the paper, a discussion section is added at the end of Section 4. For instance, we agree that designing models to explain the time delay between weak signals shifts and their consequences on the target signals is an interesting objective. We believe that this is out of the scope of the present paper but this is the topic of an ongoing work using hidden states to leverage  the impact of external signals on the main signal.\\ 


	\item {\em ``In Table 2, it would be good to clarify which methods can handle weak signals and which cannot. Then, please clarify by experiment that it is effective to construct a hybrid model by the weighted sum of two terms as in the proposed method (4).''} \medskip

	\textbf{Author response:}  In the revised version of the paper, a star is added in Table 2 and Table 4 to specify which methods have access to the weak signals and which do not. In addition, in Table 4, results of HERMES declinations with and without the weak signals are added. Finally, so as to provide more benchmarks, the Prophet method, the DeepAR model and a third HERMES variation are added for the Fashion use case.\\
	
	\item {\em ``Minor issues: i)The term "causal inference" appears in the abstract, but it is not discussed in this study and should be changed to another term. ii) The proposed method, HERMES, does not specify what it stands for. iii) Please cite Figure 1 in the text. iv) Please change the "t" in "At each time t" in section 2.1 to italic.''} \medskip

	\textbf{Author response:} We thank the referee for these remarks and the text has been corrected accordingly. Concerning the remark on the name of the proposed model, HERMES stands of Hybrid ERror-corrector Model with inclusion of External Signals.\\	
\end{itemize}
\end{document}
